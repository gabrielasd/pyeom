\documentclass{article}

\usepackage{tabularx}
\usepackage{booktabs}

\usepackage[backend=biber,style=verbose-ibid]{biblatex}
\bibliography{../refs/References.bib}


\title{CAS 741: Problem Statement\\Equation-of-motion methods}

\author{Gabriela S\'anchez D\'iaz,\\ sanchezg}

\date{}

\input{../Comments}

\begin{document}

\maketitle

\begin{table}[hp]
\caption{Revision History} \label{TblRevisionHistory}
\begin{tabularx}{\textwidth}{llX}
\toprule
\textbf{Date} & \textbf{Developer(s)} & \textbf{Change}\\
\midrule
21-09-2020 & Sanchez-Diaz & Describe the problem statement\\
28-09-2020 & Sanchez-Diaz & Clarify inputs/outputs \\
%... & ... & ...\\
\bottomrule
\end{tabularx}
\end{table}

Equation-of-motion (EOM) methods are an alternative approach to the prediction 
of molecular spectroscopic properties (excitation energies, ionization 
potentials, electron affinities, and their corresponding oscillator strengths). 
In these methods, a transition operator transfers between two stationary 
states, and their energy difference is evaluated directly. As a result of the 
transition, the number of electrons in the initial reference system and the 
final one might differ.

In addition to spectroscopic applications of the EOMs, their solutions can be 
used to approximate the transition density matrices (TDMs). TDMs are an 
important ingredient in a recent method proposed by Katarzyna 
Pernal \footcite{Pernal2018} for computing the electron correlation energy, 
which is the preeminent objective of electronic structure methods.  

This code develops a common framework for implementing and solving EOM 
equations, taking as inputs the reduced density matrices (RDMs) and the 1- and 
2-electron integrals for the reference N-electron system. The software's main 
outcomes are the energies and TDMs of an "excitation" process 
(represented by an approximated transition operator), though future extensions 
will cover the evaluation of properties related to the TDMs, such as the 
oscillator strengths and the electron correlation energy.


This tool should be useful to students and researchers in the field of 
computational quantum chemistry. It will be developed for Linux based systems, 
however, compatibility with other environments such as Windows and Mac OS 
should be possible inasmuch as it is a pure Python program using mostly 
standard libraries.

\end{document}
