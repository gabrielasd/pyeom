\documentclass{article}

\usepackage{tabularx}
\usepackage{booktabs}

\usepackage{biblatex}
\bibliography{../../refs/References}


\title{CAS 741: Problem Statement\\Equation-of-motion methods}

\author{Gabriela S\'anchez D\'iaz,\\ sanchezg}

\date{}

\input{../Comments}

\begin{document}

\maketitle

\begin{table}[hp]
\caption{Revision History} \label{TblRevisionHistory}
\begin{tabularx}{\textwidth}{llX}
\toprule
\textbf{Date} & \textbf{Developer(s)} & \textbf{Change}\\
\midrule
21-09-2020 & Sanchez-Diaz & Describe the problem statement\\
%Date2 & Name(s) & Description of changes\\
%... & ... & ...\\
\bottomrule
\end{tabularx}
\end{table}

The equation-of-motion methods (EOM) offer an alternative approach to the prediction of molecular spectroscopic properties (excitation energies, ionization potentials, electron affinities, and their corresponding oscillator strengths). In these methods, a transition operator transforms betweens two stationary states, and their energy difference gets evaluated directly. As a result of the transformation, the number of electrons in the initial reference system and the final one might differ.

In addition to the spectroscopic applications of the EOMs, their solutions can be used to approximate the transition density matrices (TDMs). TDMs are an important ingredient in a recent method proposed by Katarzyna Pernal~\footcite{Pernal2018} for determining the electron correlation energy, a key magnitude in electronic structure methods.  

The pourpose is then to develop a common framework for implementing and solving EOMs taking as inputs the reduced density matrices (RDMs) and molecular electron integrals of the reference N-electron system. 

This tool should be useful to students and researchers in the field of computational quantum chemistry. It will be developed for Linux based system, however, compatibility with other environments such as Windows and Mac OS might also be possible.

\printbibliography

\an{The citation isn't being generated correctly}

\end{document}